\section{Ejemplo: Sistema de dos estados}
Un sistema de N partículas idénticas pero distinguibles, no interactuantes, pueden adoptar dos posibles valores de energía, $+\epsilon$, $-\epsilon$. Calcular la energía libre de Helmholtz dependiente de la temperatura, la entropía y los números de ocupación. Investigar los límites de la temperatura. Usar el ensemble microcanónico.
