\section{Introducción}
En este ensayo se revisan los conceptos básicos para comenzar a entender el movimiento de las cargas y los fenómenos que originan. En temas anteriores se explicaron los conceptos de campos eléctricos tanto para conductores como para dieléctricos. Se habló del campo $\vec{\textbf{D}}$ y la polarización $\vec{\textbf{P}}$, campos importantes para comprender la naturaleza de las cargas y la teoría dipolar de la materia. Ahora, se hablará ya no solo de cargas estáticas, sino de un caso más general en donde las cargas pueden mantener un movimiento o encontrarse en reposo.