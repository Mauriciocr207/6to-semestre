\section{Ley de Ampere}
Los fenómenos magnéticos se conoces desde hace cientos de años, pues ya se había observado que ciertos materiales naturales podían atraer a otros materiales. El nombre de magnetismo se asocia con los objetos encontrados en la cercanía de la antigua ciudad de \emph{Magnesia} en el Asia Menor. A pesar de que se pensaba que la electricidad y el magnetismo eran fenómenos de distinta índole, se realizaron avances en torno al magnetismo, como la brújula. En el año de 1819, Oersted encontraría el primer indicio de que esta relación entre electricidad y magnetismo era posible. El mismo año, Oersted descubrió por accidente que una corriente eléctrica podía ejercer fuerzas sobre una brújular magentica. Más tarde, Ampere supo del trabajo de Oersted y comenzó a realizar experimentos, encontrando que dos cables con una corriente estacionaria ejercían fuerza entre ellos. Entre los años 1820 y 1825, Ampere dedujo la Ley básica de la fuerza magnética entre corrientes elécrtricas.

\subsection{Fuerza entre dos circuitos completos}

\imagen{./src/img/1.png}{Representación de dos contornos cerrados con corrientes $I$ e $I'$}{\label{fig:1}}

Obsérvese que el diagrama de la figura \ref{fig:1} no considera ninguna batería que produzca una corriente o flujo de electrones, sin embargo se puede suponer que estas baterías se encuentran situaas en alguna posición que es muy lejana y estos se pueden despreciar, considerando que se tiene un circuito completamente cerrado. Se tiene entonces, que la fuerza entre los conductores se puede expresar en función de los elementos de corriente $Id\vec{s}$ y $I'd\vec{s'}$, y la localización de estos elementos de corriente se puede expresar con $\vec{r}$ y $\vec{r}'$ y el vector de posición relativa entre estos dos vectores $\vec{R} = \vec{r} - \vec{r}'$. La ley experimental descrita por Ampere se traduce en

\begin{equation}
    F_{C' \rightarrow C}=\frac{\mu_0}{4\pi} \oint_C \oint_{C'} \frac{Id\vec{s}\times (I'd\vec{s'} \times \hat{R} ) }{R^2}
\end{equation}

La ecuación 1 se conoce como la Ley de Ampere. Esta ecuación incluye una doble integral de línea, cada una de las cuales se debe tomar sobre el circuito correspondiente. El factor $\frac{\mu_0}{4\pi}$ es un factor de proporcionalidad, donde $\mu$ es la permeabilidad del espacio libre o vacío.

