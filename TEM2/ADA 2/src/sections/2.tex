\section{Inducción magnética}
La ecuacion 1 se puede expresar también de la forma

\begin{equation}
    F_{C' \rightarrow C}=\oint_C  Id\vec{s}\times  \left(\frac{\mu_0}{4\pi} \oint_{C'} \frac{I'd\vec{s'} \times \hat{R}}{R^2} \right)
\end{equation}

Se puede observar que el factor entre paréntesis es independiente del elemento de corriente $Id\vec{s}$, por ello, se define una nueva cantidad llamada inducción magnética 

\begin{equation}
    \vec{B}(\vec{r})= \frac{\mu_0}{4\pi} \oint_{C'} \frac{I'd\vec{s'} \times \hat{R}}{R^2}
\end{equation}

Y, por lo tanto

\begin{equation}
    F_{C' \rightarrow C}=\oint_C  Id\vec{s}\times  \vec{B}(\vec{r})
\end{equation}

A la ecuación 4 se le conoce como la Ley de Biot-Savart.


